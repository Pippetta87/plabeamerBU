\documentclass[10pt,xcolor={usenames},fleqn,mathserif,serif]{beamer}

\hypersetup{pdfpagemode=FullScreen}

%% colors
\definecolor{bittersweet}{rgb}{1.0, 0.44, 0.37}
\definecolor{brilliantlavender}{rgb}{0.96, 0.73, 1.0}
\definecolor{antiquefuchsia}{rgb}{0.57, 0.36, 0.51}
\definecolor{violetw}{rgb}{0.93, 0.51, 0.93}
\definecolor{Veronica}{rgb}{0.63, 0.36, 0.94}
\definecolor{atomictangerine}{rgb}{1.0, 0.6, 0.4}
\definecolor{darkgray}{rgb}{0.66, 0.66, 0.66}
\definecolor{brightcerulean}{rgb}{0.11, 0.67, 0.84}
\definecolor{cadmiumorange}{rgb}{0.93, 0.53, 0.18}
\definecolor{ochre}{rgb}{0.8, 0.47, 0.13}
\definecolor{midnightblue}{rgb}{0.1, 0.1, 0.44}
\definecolor{lemon}{rgb}{1.0, 0.97, 0.0}
\definecolor{grey}{rgb}{0.7, 0.75, 0.71}
\definecolor{amber}{rgb}{1.0, 0.75, 0.0}
\definecolor{almond}{rgb}{0.94, 0.87, 0.8}
\definecolor{bf}{RGB}{88, 86, 88}
\definecolor{bb}{RGB}{177, 177, 177}


%%%%%%%%%%%%%%%%%%%%%%%%%%%%%%%%%%% importa pacchetti
\usepackage{usepkg}
\usepackage{beamersetup}
%%%%%%%%%%%%%%%%%%%%%%%%%%%%%%%%%%% Funzioni generali
\usepackage{functions}
%http://tex.stackexchange.com/questions/246/when-should-i-use-input-vs-include
\newcommand{\setmuskip}[2]{#1=#2\relax} %%problem usinig mu with calc (req by mathtools) loaded

\usepackage{sources}


%\usepackage{length}
%%%%%%%%%%%%%%%%%%%%%%%%%%%%%%%%%%% Funzioni per questo file main
\usepackage{mathOp}

\def\status{coazione}%ripeter
\def\keeptrying{coazione}
\usepackage{LocalF}
%%%%%%%%%%%%%%%%%%%%%%%%%%%%%%%%%

\title{Meccanica celeste (Beamer)}

% A subtitle is optional and this may be deleted
\subtitle{Moto in potenziale Newtoniano, sfera celeste, perturbazioni, determinazione orbite, ''meccanica analitica''.}

%\author{F.~Author\inst{1} \and S.~Another\inst{2}}
% - Give the names in the same order as the appear in the paper.
% - Use the \inst{?} command only if the authors have different
%   affiliation.

%\institute[Universities of Somewhere and Elsewhere] % (optional, but mostly needed)
%{
% \inst{1}
% Department of Computer Science\\
%  University of Somewhere
%  \and
%  \inst{2}%
%  Department of Theoretical Philosophy\\
%  University of Elsewhere}
% - Use the \inst command only if there are several affiliations.
% - Keep it simple, no one is interested in your street address.

\date{Febbraio 18, \today}
% - Either use conference name or its abbreviation.
% - Not really informative to the audience, more for people (including
%   yourself) who are reading the slides online

\subject{Orbite Kepleriane, effetti perturbativi, determinazione orbite da osservazioni, meccanica analitica. Moto dei 3 corpi.}
% This is only inserted into the PDF information catalog. Can be left
% out.

% Let's get started
\begin{document}

\input{tikzdir}%%contain tikz files as filecontents

\addtobeamertemplate{block begin}{\setlength\abovedisplayskip{2pt}\setlength\belowdisplayskip{2pt}\setlength\abovedisplayshortskip{2pt}\setlength\belowdisplayshortskip{2pt}}

\addtobeamertemplate{block begin}{\vspace*{-3pt}}{}
\addtobeamertemplate{block end}{}{\vspace*{-3pt}}

\begin{frame}
  \titlepage
\end{frame}

% Section and subsections will appear in the presentation overview
% and table of contents.
%\frame{\tableofcontents[onlyparts]}

\begin{frame}[label={argomenti}]{Sistemi planetari: argomenti del corso}

\tableofcontents[onlyparts]


\end{frame}

\begin{wordonframe}{Perch\'e studio queste cose?? Sviluppi; futuro.}


\end{wordonframe}


\part{Moto 2 corpi in potenziale newtoniano}\label{part:kepler}
\frame{\partpage}

\begin{frame}{this part toc}

\begin{itemize}

\item Problema ridotto: Potenziale newtoniano/ Esistenza orbite chiuse
\item costanti del moto
\item leggi di Keplero
\item Traiettoria o legge oraria
\item elementi orbitali
\item approssimazione piccola eccentricit\'a
\item determinazione orbita pianeta orbitante attorno al Sole da 3 osservabili

\end{itemize}

\end{frame}

\section{Orbite Kepleriane}

\begin{frame}{Problema ridotto}



\end{frame}

\section{Determinazione orbita da 3 osservabili (Solar system)}\label{sec:orbitobs}

\input{obs2orb}

\part{Problema Newtoniano: meccanica analitica.}\label{part:analytic}
\frame{\partpage}

\begin{frame}{this part toc}

\begin{itemize}

\item Meccanica analitica: equazione di H-J, separazione di H-J, e soluzione
\item variabili canoniche
\item (azione ridotta)
\item Teorema di Liouville-Arnold: soluzione equazioni del moto tramite quadratura
\item degenerazione periodi orbitali

\end{itemize}


\end{frame}

\section{Equazioni di Hamilton}

\begin{frame}{Hamiltoniana problema 2 corpi ridotto}

\begin{columns}

\begin{column}{0.2\textwidth}

\begin{figure}[!ht]

\includegraphics[width=\textwidth]{analytic}

\end{figure}


\end{column}

\begin{column}{0.8\textwidth}

\begin{align*}
&T=\frac{1}{2}\mu(\dot{r}^2+r^2\dot{\beta}^2+r^2\cos{\beta}^2\dot{\lambda}^2)\\
&V=-k^2\frac{\mu}{r}\\
&L=T-V
\end{align*}

\end{column}

\end{columns}

\begin{block}{Momenti coniugati}

\begin{align*}
&p_r=\PDy{\dot{r}}{L}=\mu\dot{r}\\
&p_{\beta}=\PDy{\dot{\beta}}{L}=\mu r^2\dot{\beta}\\
&p_{\lambda}=\PDy{\dot{\lambda}}{L}=\mu r^2\cos{\beta}^2\dot{\lambda}\ \dot{p}_{\lambda}=0
\end{align*}

\end{block}

\begin{block}{Hamiltoniana}

\begin{align*}
&H(p,q)=\frac{1}{2\mu}(p_r^2+\frac{1}{r^2}p_{\beta}^2+\frac{1}{r^2\cos{\beta}^2}p_{\lambda}^2)-\frac{k^2\mu}{r}
\end{align*}


\end{block}

\end{frame}

\begin{wordonframe}{Meccanica Hamiltoniana}

\begin{block}{Equazioni di Hamilton}
\begin{align*}
&\dot{q}=\PDy{p}{H}\\
&\dot{p}=-\PDy{q}{H}
\end{align*}
\end{block}

$\TDy{t}{H}=\PDy{t}{H}$

\end{wordonframe}


\section{Equazione di Hamilton-Jacobi}

\begin{block}{Trasformazioni canoniche}

\begin{align*}
(p,q)\to(\xi(p,q),\eta(p,q))
\end{align*}

che lasciano invariate le equazioni di Hamilton.

\end{block}

\begin{block}{Problema di H-J}

Trovare trasformazione canonica tale che $K=0$: $(\xi,\eta)$ sono costanti del moto.

\end{block}

\begin{block}{Funzione generatrice $S(q,\eta,t)$}

\begin{align*}
&\xi=\PDof{\eta}S(q,\eta,t)\ (2: \xi(p,q))\\
&p=\PDof{q}S(q,\eta,t)\ (1:\eta(q,p))\\
&K=H+\PDy{t}{S}\ K=0\Leftrightarrow H(q,\PDy{q}{S},t)=-\PDy{t}{S}
\end{align*}


\end{block}


\section{Sistemi integrabili e degenerazione}




\part{Attrazione non Newtoniana, orbite non Kepleriane: perturbazioni.}\label{part:perturbation}
\frame{\partpage}

\begin{frame}{this part toc}

\begin{itemize}

\item Potenziale non-Newtoniano: perturbazioni orbite non Kepleriane
\item Sviluppi in multipoli dell'energia potenziale
\item Moto perturbato
\item Metodo di Newton
\item Precessione luni-solare
\item Perturbazione degli elementi orbitali: meccanica analitica.
\item Perturbazioni periodiche e secolari
\item Analogo quantistico
\item Elementi osculanti

\end{itemize}

\end{frame} 

\section{Primario non sferico}

\begin{frame}{Sviluppo in multipoli energia potenziale}

\begin{align*}
V(\vec{r})=-\frac{Gm}{r}[\int\rho'\,d^3r'+\cos{\theta}\int\frac{r'}{r}\cos{\theta'}\rho(\vec{r}')\,d^3r'\\
+\frac{1}{2}(3\cos{\theta}^2-1)\int\frac{r'}{r}\frac{1}{2}(3\cos{\theta}^2-1)\rho(\vec{r}')\,d^3r'+\ldots]
\end{align*}

\begin{block}{Potenziale di quadrupolo}

\begin{equation*}
V(\vec{r})=-\frac{Gm}{r}[M+\frac{1}{2r^2}(A-C)(3\cos{\theta}^2-1)+\ldots]
\end{equation*}

\end{block}

\end{frame}

\begin{wordonframe}{Momento inerzia e $J_2$}

\begin{align*}
&I_x=I_y=A=\int(x'^2+z'^2)\rho\,d^3r'=\int(y'^2+z'^2)\rho\,d^3r'\\
&I_z=C=\int(x'^2+y'^2)\rho\,d^3r'\\
&J_2=\frac{C-A}{MR^2}\, GmM=k^2\mu\\
\end{align*}

$J_2>0$ for any planet flatten by rotation

\begin{block}{Terra-Luna}

\begin{align*}
J_2^{\oplus}\approx\num{e-3}\\
(\frac{R}{r})^2\approx(\frac{1}{60})^2\approx\num{3e-4}\\
\frac{V_Q}{V_N}\approx\num{3e-7}
\end{align*}

\end{block}

\end{wordonframe}

\begin{frame}{Forza perturbatrice}

\begin{columns}
\begin{column}{0.6\textwidth}

\begin{align*}
&V=-\frac{k^2\mu}{r}+V'\\
&V'=-\frac{3Q}{r^5}(\frac{1}{3}-\cos{\theta}^2)\\
&Q=\frac{1}{2}k^2\mu R^2J_2
\end{align*}

\end{column}
\begin{column}{0.4\textwidth}

\begin{figure}[!ht]
\includegraphics[width=\textwidth]{perturbation}
\end{figure}

\end{column}
\end{columns}

$e$, $i$, $|\vec{J}|$ non variano in media su un periodo.

\begin{block}{Periodo del moto radiale (i=0)}

Traiettoria a ''rosetta'':

\begin{equation*}
T_{\lambda}=\frac{\pi}{\Delta}T_{r}
\end{equation*}

\end{block}

\end{frame}

\begin{wordonframe}{forza perturbatrice}

\begin{align*}
&f_x=3Qr\expy{-4}(5\cos{\theta}^2-1)\sin{\theta}\cos{\lambda}\\
&f_y=3Qr\expy{-4}(5\cos{\theta}^2-1)\sin{\theta}\sin{\lambda}\\
&f_z=3Qr\expy{-4}(5\cos{\theta}^2-1)\cos{\theta}
\end{align*}

\begin{block}{Integrali primi del moto ($i=0$)}
\begin{align*}
&\mu r^2\dot{\lambda}^2=J\\
&\frac{1}{2}\mu\dot{r}^2-\frac{k^2\mu}{r}+\frac{J^2}{2\mu r^2}+V'=E
\end{align*}
\end{block}

\begin{align*}
&T_r=2\int_{r_{min}}^{r_{max}}\,dr(\frac{2E}{\mu}-\frac{2V'}{\mu}-\frac{J^2}{\mu^2r^2}+\frac{2k^2}{r})\expy{-\frac{1}{2}}\\
&\Delta=\frac{J}{\mu}\int_{r_{min}}^{r_{max}}\,dr(\frac{2E}{\mu}-\frac{2V'}{\mu}-\frac{J^2}{\mu^2r^2}+\frac{2k^2}{r})\expy{-\frac{1}{2}}
\end{align*}

\end{wordonframe}

\begin{frame}{Effetti su orbita}

\begin{itemize}
\item Linea dei nodi regresses for prograde orbit ($i\neq\pi/2$) - Piano orbita ruota backward

\item $\forall i<\ang{46.5}$ il pericentro ruota in avanti.

\item $\forall i<\sin{\sqrt{\frac{2}{3}}}\expy{-1}=\ang{54.7}$ mean anomaly increases at rate greater than n: orbital period is less then for keplerian orbit about spherical planet of same mass.

\item We can determine $J_2$ for planets having satellites in eccentric/inclined orbit.

\end{itemize}

\end{frame}

\section{Metodo di Newton}

f forza centrale

\begin{equation*}
\mu\ddot{r}=f+\frac{c}{r^3}+\frac{J^2}{\mu r^3}=f+\frac{J'^2}{\mu r^3}
\end{equation*}

Stessa moto radiale con diverso j; moto angolare perturbato ha velocit\'a proporzionale a quello imperturbato: $\frac{\dot{\lambda}}{\dot{\lambda'}}=\frac{J}{J'}$.

\begin{block}{Orbita quasi-circolare}
Moto medio del pericentro: $\frac{3}{2}n\frac{R^2}{a^2}J_2$.
\end{block}

\begin{wordonframe}{Perturbazione orbita quasi-circolare}

\begin{align*}
V'(\theta=\pi/2)=-\frac{3Q}{r^4}\approx\frac{3Q}{r^2a^2}-\frac{6Q}{ar^3}+o((r-a)^2)
\end{align*}

\end{wordonframe}

\section{Precessione luni-solare}

$\leftmoon$

\section{Perturbazion}


\part{Orbite non Kepleriane: problema dei 3 corpi.}\label{part:threebody}
\frame{\partpage}

\begin{frame}{this part toc}

\begin{itemize}

\item Problema dei 3 corpi ristretto

\item criterio di Tisserand

\item Problema generale dei 3 corpi

\item comportamento caotico

\item Disuguaglianza di Easton

\end{itemize}


\end{frame}


\input{threebody}


\end{document}