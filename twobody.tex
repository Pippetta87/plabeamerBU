\section{Orbite Kepleriane}

\begin{frame}{Problema ridotto}

\begin{columns}

\begin{column}{0.55\textwidth}

\input{reducedproblem}

\begin{align*}
%&m_1\ddvec{r}_1=\frac{m_1m_2G}{r^3}\vec{r}\\
%&m_2\ddvec{r}_2=\frac{m_1m_2G}{r^3}\vec{r}\\
&\mu\ddvec{r}=-k^2\frac{\vec{r}}{r^3}=-\frac{\gamma}{\mu}\frac{\vec{r}}{r^3}

\end{align*}

\end{column}

\begin{column}{0.45\textwidth}

\end{column}

\end{columns}

\end{frame}

\begin{wordonframe}{Definizione problema 2 corpi attrazione newtoniana}

\begin{align*}
&\vec{R}=\frac{m_1\vec{r}_1+m_2\vec{r}_2}{M}\\
&\vec{r}=\vec{r}_1-\vec{r}_2
\end{align*}

\end{wordonframe}


\begin{frame}{Vettori conservati}

\begin{columns}

\begin{column}{0.55\textwidth}

\input{conservedvector}

\end{column}

\begin{column}{0.45\textwidth}

\end{column}

\end{columns}

\end{frame}

\begin{frame}{Elementi orbitali}

\end{frame}


\begin{frame}{Orbite Keplerine}

\end{frame}

\begin{frame}{Approx. piccola eccentricit\'a}

\end{frame}

