\section{Orbite Kepleriane}

\begin{frame}{Problema ridotto}

\begin{columns}

\begin{column}{0.55\textwidth}

\input{reducedproblem}


\end{column}

\begin{column}{0.45\textwidth}

\begin{align*}
&m_1\ddvec{r}_1=\frac{m_1m_2G}{r^3}\vec{r}\\
&m_2\ddvec{r}_2=\frac{m_1m_2G}{r^3}\vec{r}\\
&\mu\ddvec{r}=-k^2\frac{\vec{r}}{r^3}=-\frac{\gamma}{\mu}\frac{\vec{r}}{r^3}
\end{align*}

\end{column}

\end{columns}

\begin{align*}
&E_T=\frac{1}{2}m_1\dvec{r}_1^2+\frac{1}{2}m_2\dvec{r}_2^2-G\frac{m_1m_2}{r}\\
&=E_{CM}+E=\frac{1}{2}M\dvec{R}^2+\frac{1}{2}\mu\dvec{r}^2-G\frac{M\mu}{r}
\end{align*}


\end{frame}

\begin{wordonframe}{Definizione problema 2 corpi attrazione newtoniana}

\begin{align*}
&\vec{R}=\frac{m_1\vec{r}_1+m_2\vec{r}_2}{M}\\
&\vec{r}=\vec{r}_1-\vec{r}_2
\end{align*}

\end{wordonframe}


\begin{frame}{Moto piano}

\begin{columns}

\begin{column}{0.55\textwidth}

\input{conservedvector}

\end{column}

\begin{column}{0.45\textwidth}

\begin{align*}
&\vec{J}=\mu\vec{r}\wedge\dvec{r}\\
&\vec{J}\cdot\vec{r}=0\\
&\vec{J}\cdot\dvec{r}=0
\end{align*}

\begin{align*}
&\vec{L}=\dvec{r}\wedge\vec{J}-\gamma\frac{\vec{r}}{r}=\dvec{r}\wedge\vec{J}-Gm_1m_2\frac{\vec{r}}{r}\\
&\dvec{L}=0\\
&\scap{J}{L}=0\\
&\scap{L}{r}=\frac{1}{\mu}J^2-\gamma r\\
&L^2=
\end{align*}

$\vec{r}$ ruota in piano perpendicolare a $\vec{J}$ e $\vec{L}$ \'e sul piano dell'orbita.


\end{column}

\end{columns}


\end{frame}

\begin{frame}{Elementi orbitali}

\end{frame}


\begin{frame}{Orbite Keplerine}

\end{frame}

\begin{frame}{Approx. piccola eccentricit\'a}

\end{frame}

