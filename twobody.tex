\section{Orbite Kepleriane}

\begin{frame}{Problema ridotto}

\begin{columns}

\begin{column}{0.3\textwidth}

\input{reducedproblem}

\begin{align*}
&m_1\ddvec{r}_1=\frac{m_1m_2G}{r^3}\vec{r}\\
&m_2\ddvec{r}_2=\frac{m_1m_2G}{r^3}\vec{r}\\
&\mu\ddvec{r}=-k^2\frac{\vec{r}}{r^3}=-\frac{\gamma}{\mu}\frac{\vec{r}}{r^3}
\end{align*}


\end{column}

\begin{column}{0.7\textwidth}

\begin{block}{Costanti del moto}

\begin{align*}
&E_T=\frac{1}{2}m_1\dvec{r}_1^2+\frac{1}{2}m_2\dvec{r}_2^2-G\frac{m_1m_2}{r}\\
&=E_{CM}+E=\frac{1}{2}M\dvec{R}^2+\frac{1}{2}\mu\dvec{r}^2-G\frac{M\mu}{r}
\end{align*}

\begin{align*}
&\vec{J}=\mu\vec{r}\wedge\dvec{r}\\
&\vec{J}\cdot\vec{r}=0\\
&\vec{J}\cdot\dvec{r}=0
\end{align*}

\begin{align*}
&\vec{L}=\dvec{r}\wedge\vec{J}-\gamma\frac{\vec{r}}{r}=\dvec{r}\wedge\vec{J}-Gm_1m_2\frac{\vec{r}}{r}\\
&\dvec{L}=0\\
&\scap{J}{L}=0\\
&\scap{L}{r}=\frac{1}{\mu}J^2-\gamma r\\
&L^2=\gamma^2+\frac{2}{\mu}EJ^2
\end{align*}

\end{block}

\end{column}

\end{columns}

\end{frame}

\begin{wordonframe}{Definizione problema 2 corpi attrazione newtoniana}

\begin{align*}
&\vec{R}=\frac{m_1\vec{r}_1+m_2\vec{r}_2}{M}\\
&\vec{r}=\vec{r}_1-\vec{r}_2
\end{align*}

$\vec{r}$ ruota in piano perpendicolare a $\vec{J}$ e $\vec{L}$ \'e sul piano dell'orbita.

\end{wordonframe}


\begin{frame}{forma dell'orbita}

\begin{columns}

\begin{column}{0.4\textwidth}

\input{conservedvector}

\begin{block}{Periodo orbitale}

\begin{align*}
&JT=2\mu\pi ab\\
&T=2\pi\sqrt{\frac{a^3}{k^2}}
\end{align*}

\end{block}

\end{column}

\begin{column}{0.45\textwidth}

\begin{align*}
&Lr\cos{v}+\gamma r=\frac{1}{\mu}J^2\\
&r=\frac{p}{1+e\cos{v}}\\
&p=\frac{J^2}{\gamma\mu},\ e=\frac{L}{\gamma}
\end{align*}

\begin{block}{Orbite ellittiche}

\begin{align*}
&E<0,\ e<1\\
&r_{min}=\frac{p}{1+e},\ r_{max}=\frac{p}{1-e}\\
&a=\frac{1}{2}(r_{min}+r_{max})=\frac{p}{1-e^2}=-\frac{\gamma}{2E}\\
&J_{max}=\gamma\sqrt{\frac{\mu}{2|E|}}
\end{align*}

\end{block}

\end{column}

\end{columns}

\end{frame}

\begin{wordonframe}{Orbita kepleriana}

L'orbita \'e una conica con asse lungo $\vec{L}$, il modulo di L dipende da E e J

\begin{align*}
&e\gtreqless1\Leftrightarrow L\gtreqless\gamma\Leftrightarrow E\gtreqless0\\
&a=\frac{\gamma}{2|E|}\\
&b/a=\sqrt{1-e^2}
\end{align*}

connessione tra aspetti geometrici e dinamici: $a=-\frac{\gamma}{2E}$.

Per una certa energia E J non pu\'o superare quel valore per cui $a=b$.

\begin{align*}
&dS=\frac{1}{2}rr\dot{\theta}\,dt\\
&J=2\mu\TDy{t}{S}
\end{align*}

Terza legge di Keplero: $T\propto a\expy{\frac{3}{2}}$ (per pianeti maggiori sistema solare $msun{}/m\approx\num{e-3}$)

moto medio $\frac{2\pi}{T}$: $n^2a^3=k^2$

\end{wordonframe}


\begin{frame}{Elementi orbitali}

\input{orbelements}

\end{frame}


\begin{frame}{Orbite Kepleriane}

\end{frame}

\begin{frame}{Approx. piccola eccentricit\'a}

\end{frame}

