
%%%Appunti penco
\part{Dinamica/Meccanica celeste}

\chapter{Problema a 2 corpi}
\PartialToc

\section{Problema dei 2 corpi}

\subfile{tikz/reducedproblem.tex}

\begin{usefull}{Costante di Gravitazione universale: G}

\begin{equation*}
G\approx4.302*10^{-3}\,pc/M_{\odot}^{3}(km/s)^2
\end{equation*}

\end{usefull}

\subsection{Problema ridotto}


\subsubsection{Legge di gravitazione}

La forza che agisce sui 2 corpi \'e risp
\begin{align*}
\vec{F_m}=-GMm\frac{\vec{r}}{r^3}\\
\vec{F_M}=-GMm\frac{\vec{r}}{r^3}\\
\end{align*}

\begin{definition}{Coefficiente $\gamma$ del problema a 2 corpi}

Definisco $\gamma=GmM$. G \'e la costante di gravitazione universale.

\end{definition}


Ridurremo il problema a 2 corpi al problema di un corpo di massa $\mu$ attratto da un centro $O$
\begin{align*}
m\ddvec{r_1}=\vec{F_m}=-GMm\frac{\vec{r}}{r^3}\\
M\ddvec{r_2}=\vec{F_M}=-GMm\frac{\vec{r}}{r^3}\\
\end{align*}
dividendo e sottraendo si perviene all'equazione fondamentale

\begin{align}
&\mu\ddvec{r}=-\gamma\frac{\vec{r}}{r^3}\label{eq:fondamentale}\\
&\ddvec{r}=-k^2\frac{\vec{r}}{r^3}
\end{align}

Introduco la costante di Gauss\index{Costante di Gauss}
\begin{align*}
\frac{1}{\mu}=\frac{1}{M}+\frac{1}{M}\\
k^2=\frac{\gamma}{\mu}=G(M+m)
\end{align*}

\subsection{Costanti del moto: energia, momento angolare, vettore di Lenz}
Avendo come obiettivo di integrare ~\ref{eq:fondamentale} elenco le costanti del moto (Per primi due vedi moto a 2 corpi riferito al CM)

\begin{enumerate*}

\item Energia\\
\begin{equation*}
E=T+V=\frac{1}{2}\mu|\dot{r}|^2-\frac{\gamma}{r}
\end{equation*}

\item Momento angolare\\
\begin{equation*}
\vec{J}=\mu\vecp{r}{\dot{r}}
\end{equation*}
quindi \lbt{\scap{J}{r}=0}{\scap{J}{\dot{r}}=0} cio\'e il momento angolare \'e perpendicolare al vettore posizione e alla velocit\'a.


\item Vettore di Lenz\\
\begin{equation*}\label{eq:lenzv}
\vec{L}=\vecp{\dot{r}}{J}-\gamma\frac{\vec{r}}{r}
\end{equation*}


\subfile{tikz/lenzvector.tex}

\end{enumerate*}

Esplicitando $\vec{\dot{L}}$ e usando la conservazione dell'energia e l'equazione del moto ~\ref{eq:fondamentale} vedo che $\vec{L}$ \'e costante del moto.

\subsection{Propriet\'a del vettore di Lenz}

\begin{itemize*}
\item Ortogonale al mometno angolare
\begin{equation*}\label{eq:JLzero}
\scap{J}{L}=0
\end{equation*}

\item Modulo del vettore di Lenz
\begin{equation*}\label{eq:moduloL}
L^2=\gamma^2+\frac{2}{\mu}EJ^2
\end{equation*}

\item Proiezione di $\vec{L}$ lungo $\vec{r}$
\begin{equation*}\label{eq:Lr}
\scap{L}{r}+\gamma r=\frac{1}{\mu}J^2
\end{equation*}
\end{itemize*}

\subsection{Conseguenze delle leggi di conservazione}

\begin{enumerate*}
\item Moto piano\\
Esiste un vettore $\vec{J}$, costante del moto, sempre ortogonale ad $\vec{r}$: $\vec{r}$ ruota in un piano ortogonale a $\vec{J}$.
\item Il vettore di Lenz \'e sul piano dell'orbita\\
Segue da ~\ref{eq:JLzero}
\end{enumerate*}


\section{Forma dell'orbita}


\subsubsection{Anomalia vera}

Chiamo anomalia vera l'angolo tra $\vec{L}$ e $\vec{r}$.

Utilizzo ~\ref{eq:Lr} per ricavare l'equazione dell'orbita $r=\frac{\frac{J^2}{\mu}}{\gamma+L\cos{v}}$ e ponendo \lbt{p=\frac{J^2}{\gamma\mu}}{e=\frac{L}{\gamma}} ottengo la conica
\begin{equation*}\label{eq:orbitaconica}
r=\frac{p}{1+e\cos{v}}
\end{equation*}

\begin{itemize*}
\item $e>1$: Iperbole.

Deve essere $L>0\Rightarrow E=T+V>0$ e poich\'e \lbt{V<0}{V\abc{r\to\infty}{0}} il corpo pu\'o arrivare all'infinito con una certa energia cinetica

\item $e=1$: parabola.

Il corpo pu\'o andare all'infinito con velocit\'a tendente a zero.

\item $e<1$: ellisse.

Deve essere $L<\gamma\Rightarrow E<0$ (segue da ~\ref{eq:moduloL}): il corpo non pu\'o andare all'infinito dato che si avrebbe $T<0$.

\end{itemize*}

\subsection{Significato fisico del vettore di Lenz}

L'asse della conica \'e lungo $\vec{L}$; il modulo non ci da altre informazioni dato che dipende da $E$ e da $|\vec{J}|$ 



\section{Leggi di Keplero}



\section{Soluzione del moto}
Cenno i sistemi di coordinate Equatoriali ed Eclittiche - Soluzione del moto: anomalia vera, eccentrica, media - Approssimazioni per piccole eccentricit\'a - 

\section{ degli elem. orbitali da tre osservazioni (metodo di Laplace) }
Elementi dell'orbita di un pianeta (Tabella del sistema solare) - Determinazione degli elem. orbitali da tre osservazioni (metodo di Laplace) 

\section{effemeridi}
Riepilogo del problema dei due corpi - Calcolo delle effemeridi


\section{Stelle doppie.}
\begin{todo}{roche surface}
stellar interior: pg 107-109
\end{todo}


\chapter{Problema dei 3 corpi}


Problema ristretto dei tre corpi: Riferimento rotante, Integrale di Jacobi, superfici di Hill - Punti di Lagrange 
\PartialToc

\section{Riferimento rotante}

Determinazione e stabilit\'a dei punti Punti di Langrange - Cenno alla stabilit\'a delle config. gerarchiche - Asteroidi Troiani e Greci - Problema delle comete - Criterio di Tisserand - Elementi e valori numerici del sistema Sole-Terra-Luna.
Campo delle forze di marea - Sistema Terra-Luna-Sole - Cenno alle maree terrestri
Perturbazione da primario non sferico: campo di forze aggiuntivo - Effetto sul periodo, sul nodo ascendente, sulla posizione del perigeo. - Precessione lunisolare - 


\stopcontents[chapters]


\backmatter

\part{Appendice}

\chapter{Programma Astrofisica/a 2016}

\begin{itemize}
    \item equilibrio idrostatico: sistemi autogravitanti.
    
Sistemi autogravitanti; l'equazione dell'equilibrio idrostatico. La scala di tempo dei processi dinamici: il tempo di free fall.

\item Modelli politropici e applicazioni: l'equazione di eddington e il peso della pressione di radiazione sull'equilibrio di una stella. 

Modelli politropici di nane bianche e stelle di neutroni. Il ruolo degli effetti quantistici sull'equazione di stato. La neutronizzazione della materia.

\item Corpi autogravitanti in rotazione.

Ellissoidi di Mc laurin e di Jacobi. 

\item Energia e stabilita' di una struttura autogravitante: il teorema del viriale (richiami); il tempo di Kelvin-Helmoltz. Sistemi a calore specifico negativo.
il criterio di stabilita' e gli indici adiabatici.

\item Meccanismi di produzione e trasporto dell'energia.

Il trasporto radiativo e il gradiente di temperatura. Il limite di Eddington e la relazione massa luminosita'. 

\item Trasporto convettivo.

Il criterio di Schwarzschild; convezione adiabatica e superadiabatica (cenni).

\item La radiazione stellare; effetti di assorbimento interstellare e dovuti all'atmosfera. osservazioni nel visibile: curve di sensibilita' tradizionali. 

\item Sistemi fotometrici e indici di colore. Gli spettri stellari; classificazione degli spettri stellari; la temperatura efficace. La classe di luminosit\'a.

\item Righe di assorbimento negli spettri stellari: stima dell'intensit\'a in funzione dell'abbondanza nello stato di partenza. larghezza equivalente. Allargamento delle righe (Doppler, per pressione, per rotazione).

\item Utilizzo degli spettri stellari a diverso livello di dispersione. Problemi di distanza in astrofisica: le distanze stellari: la parallasse (diurna e annua): introduzione.

\item Parallassi stellari; definizione di parsec. Moti propri e stime di distanza basate sul moto proprio. Il diagramma di Hertzsprung-Russell delle stelle parallassate 

\item Interpretazione del diagramma di H-R. Diagrammi di ammasso. Passaggio a raggi e luminosita'. Le subnane.

\item Metodi di misura di masse e raggi: le stelle doppie. Caratterizzazione delle stelle doppie visuali, spettroscopiche,a eclisse. 

\item Determinazione di masse e raggi in sistemi binari. relazioni massa raggio e massa-luminosita'. Criteri di classificazione delle stelle per popolazione: gli oggetti piu' brillanti in ciascun sistema, il comportamento cinematico, le peculiarita' spettroscopiche. 

\item Popolazioni stellari e composizione chimica delle stelle. 

\item Reazioni nucleari nelle stelle; generalita'; i cicli PP e CNO; la reazione di fusione del carbonio. Reazioni successive 

\item Il Sistema Solare: scoperta e osservazione di corpi planetari. I pianeti. e pianeti extrasolari.

Pianeti nani, satelliti e anelli, asteroidi (introduzione). 

Asteroidi (conclusione), TNO e comete.

Scoperta di pianeti extrasolari: tecniche principali (spettroscopia, transiti) e altri metodi (imaging, microlensing, pianeti intorno alle pulsar).

Le caratteristiche dei pianeti extrasolari; proprieta' generali. Pianeti in zone abitabili 

\item universo extra-galattico.

L'universo extragalattico: nebulose e galassie: la scoperta delle Cefeidi extragalattiche; criteri di distanza in ambito extragalattico. Classificazione delle galassie: il diagramma di Hubble.

\item Caratteristiche spettroscopiche delle galassie. Distribuzione spaziale della luminosita'. Quasar e nuclei galattic attivi (cenni). Oggetti ad alto redshift.

\item La legge di Hubble. Modelli di universo (cenni). Il Big Bang e i problemi del modello standard. La radiazione di fondo. 

\item Evoluzione del modello standard: l'inflazione, la dark matter e la dark energy. La cosmologia come potenziale laboratorio di fisica fondamentale (cenni). 

\item Elementi di meccanica celeste: il problema dei due corpi e gli elementi orbitali.

Mar 24/11/2015 09:00-11:00 (2:0 h) lezione: Il problema dei tre corpi: generalita'; il problema ristretto. I punti lagrangiani; i "Troiani". Introduzione all'invariante di Tisserand. (Paolo Paolicchi)
Gio 26/11/2015 09:00-11:00 (2:0 h) lezione: L'invariante di tisserand e la classificazione delel comete. Problemi di integrabilita' di sistemi a tre o piu' corpi. Fenomeni caotici (cenni). Criteri per sistemi a piu' corpi: la disuguaglianza di Easton e la stabilita' "gerarchica". (Paolo Paolicchi)
Lun 30/11/2015 11:00-13:00 (2:0 h) lezione: Approccio perturbativo al moto in sistemi complessi. Il metodo di Gauss e la stabilita' del Sistema Solare (cenni). (Paolo Paolicchi)
Mar 01/12/2015 09:00-11:00 (2:0 h) lezione: Perturbazioni dovute a corpi interagenti non sferici. Calcolo approssimato del moto del pericentro. Moto di precessione dell'asse di rotazione terrestre (cenni); le maree e la sincronizzazione parziale o totale dei moti di rotazione (cenni). (Paolo Paolicchi)
Gio 03/12/2015 09:00-11:00 (2:0 h) lezione: Processi di formazione; quadro generale della formazione stellare e planetaria; collasso "isotermo" e "adiabatico"; ruolo della rotazione. (Paolo Paolicchi)
Gio 10/12/2015 09:00-11:00 (2:0 h) lezione: Il criterio di Jeans e le sue generalizzazioni (turbolenza, campi magnetici, rotazione uniforme e differenziale). (Paolo Paolicchi)
Lun 14/12/2015 11:00-13:00 (2:0 h) lezione: Il problema delle instabilita' in disco sottile, e idee generali in merito alla formazione dei sistemi planetari. La struttura dei sistemi planetari. (Paolo Paolicchi)

\end{itemize}

\clearpage