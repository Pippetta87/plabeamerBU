

%% document-wide tikz options and styles

\tikzset{%
  >=latex, % option for nice arrows
  inner sep=0pt,%
  outer sep=2pt,%
  mark coordinate/.style={inner sep=0pt,outer sep=0pt,minimum size=4pt,
  fill=black,circle},%
	sundot/.style={
	fill, color=yellow, circle, inner sep=3.5pt}
}


\begin{tikzpicture} % "THE GLOBE" showcase


    \def\R{1.5} % sphere radius
    \def\angEl{5} % elevation angle
    \def\angAz{105} % azimuth angle
    \def\angPhi{-40} % longitude of point P
    \def\angBeta{19} % latitude of point P

    \pgfmathsetmacro\H{\R*cos(\angEl)} % distance to north pole
    \tikzset{xyplane/.style={cm={cos(\angAz),sin(\angAz)*sin(\angEl),-sin(\angAz),
                                  cos(\angAz)*sin(\angEl),(0,-\H)}}}
    \LongitudePlane[xzplane]{\angEl}{\angAz}
    \LatitudePlane[equator]{\angEl}{0}

     \filldraw[ball color=white, fill opacity=1] (0,0) circle (\R);
    \draw (0,0) circle (\R);
    \coordinate (O) at (0,0);
    \coordinate[mark coordinate] (N) at (0,\H);
    \coordinate[mark coordinate] (S) at (0,-\H);
%		\coordinate (horizon) at (-30.0:\R);
%		\coordinate (equator) at (\H,0);


    \draw[->] (0,-\H-5) -- (0,\R+5) node[above] {}; %axis of rotation
%		\draw[->,rotate=-30.0] (0,-\H-5) -- (0,\R+5) node[above] {\bf{Zenith}}; %axis of rotation

    \path[xzplane] (\R,0) coordinate (XE);

%		\DrawLatitudeCircle[\R,color=blue]{0} % equator
%		\DrawLatitudeCircle[\R,rotate=23.5,color=red]{0}
%		\node[right,color=red] at (horizon) {\bf{Horizon}};
%		\node[above=9pt, right=-5pt, color=blue] at (equator) {\bf{Equator}};
%		\node[label={above right:\bf{Equator}}] at (equator) {};

		\path (N) node[align=left, above=1.5em, right] {\textbf{North}\\ \textbf{pole}} (N);
		\path (S) node[align=left, below=1.5em, right] {\textbf{South}\\ \textbf{pole}} (S);
%    \node[above=10pt, left=3pt] at (N) {\bf{North}};
%    \node[above=2pt, left=12pt] at (N) {\bf{pole}};

%    \node[below=5pt, right=4pt] at (S) {\bf{South}};
%    \node[below=14pt, right=4pt] at (S) {\bf{pole}};

    \def\R{6} % sphere radius
    \def\angEl{5} % elevation angle
    \def\angAz{105} % azimuth angle
    \def\angPhi{-40} % longitude of point P
    \def\angBeta{19} % latitude of point P

    \pgfmathsetmacro\H{\R*cos(\angEl)} % distance to north pole
    \tikzset{xyplane/.style={cm={cos(\angAz),sin(\angAz)*sin(\angEl),-sin(\angAz),
                                  cos(\angAz)*sin(\angEl),(0,-\H)}}}
    \LongitudePlane[xzplane]{\angEl}{\angAz}
    \LatitudePlane[equator]{\angEl}{0}

    \filldraw[ball color=white, fill opacity=0.3] (0,0) circle (\R);
    \draw (0,0) circle (\R);

    \coordinate (O) at (0,0);
    \coordinate[mark coordinate] (N) at (0,\H);
    \coordinate[mark coordinate] (S) at (0,-\H);
		\coordinate (Equator) at (\H,0);
		\coordinate (Ecliptic) at (23.5:\R);
    \path[xzplane] (\R,0) coordinate (XE);

		\draw[->,ultra thick] (0:\R) arc (0:23.5:\R);
		\coordinate (tilt) at (11.75:\R);
		\node[right=5pt] at (tilt) {$23.5^{\circ}$};
%		\coordinate (obslat) at (75:0.4*\R);
%		\node[above=5pt] at (obslat) {$30^{\circ}$};
		\DrawLatitudeCircleName[\R,color=blue]{0}{celeq} % equator
%		\draw 
		\DrawLatitudeCircleName[\R,rotate=23.5, color=orange]{0}{ecl} % ecliptic
		\node[right=0.5em, color=blue] at (Equator) {\textbf{Celestial equator}};
		\node[above=0.5em, right=0.5em, color=orange] at (Ecliptic) {\textbf{Ecliptic}};
		\path[name intersections={of= celeq and ecl,by={vereq}}];%, [label=above left:\textbf{Autumnal equinox}]b}}];
		\coordinate[mark coordinate] () at (vereq) {};
		\path (vereq) node[align=right, below=1.3em, right] {\textbf{Vernal equinox}\\ March 20th}
		(vereq);
		\coordinate[mark coordinate] (sumsol) at (23.5:\R);
		\path (sumsol) node[align=right, above=1.3em, left=1em] {\textbf{Summer solstice}\\ June
		21st} (sumsol);
		\coordinate[mark coordinate] (winsol) at (23.5+180:\R);
		\path (winsol) node[align=right, below=1.5em, right=0.5em] {\textbf{Winter solstice}\\
		December 21st} (winsol);
		\coordinate[mark coordinate] (auteq) at (100.5:0.09*\R);
		\path (auteq) node[align=right, above=1.3em, left=0.5em] {\textbf{Autumnal equinox}\\
		September 23rd} (auteq);
		\coordinate[sundot] (sun) at (15:0.5*\R);
		\draw[->,very thick] (sun) -- (18.4:0.7*\R);
		\node[above=1.0em,right=0.1em] at (vereq) {$\aries$};
		\node[below=1.0em] at (sun) {\textbf{The sun}};
%		\draw[->] (auteq) -- (120:0.3*\R);
%		\node[mark coordinate,align=center,below right] at (a) {Vernal \\ equinox};
%		\node[mark coordinate,label=above left:\textbf{Summer solstice}]
%		\node[mark coordinate] at (a) {};
%		\coordinate[mark coordinate] (a) at (intersection-2);
%		\node[label={above right:\bf{Celestial horizon}},color=red] at (Horizon) {};
%		\node[label={right:\bf{Celestial equator}},color=red] at (Equator) {};

%		\DrawLongitudeCircle[\R]{\angAz+15}{} % xzplane

    \node[above=7pt, left=5pt] at (N) {\bf{North celestial pole}};
    \node[below=8pt, right=5pt] at (S) {\bf{South celestial pole}};

\end{tikzpicture}