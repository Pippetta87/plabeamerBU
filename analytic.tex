\section{Equazioni di Hamilton}

\begin{frame}{Hamiltoniana problema 2 corpi ridotto}

\begin{columns}

\begin{column}{0.2\textwidth}

\begin{figure}[!ht]

\includegraphics[width=\textwidth]{analytic}

\end{figure}


\end{column}

\begin{column}{0.8\textwidth}

\begin{align*}
&T=\frac{1}{2}\mu(\dot{r}^2+r^2\dot{\beta}^2+r^2\cos{\beta}^2\dot{\lambda}^2)\\
&V=-k^2\frac{\mu}{r}\\
&L=T-V
\end{align*}

\end{column}

\end{columns}

\begin{block}{Momenti coniugati}

\begin{align*}
&p_r=\PDy{\dot{r}}{L}=\mu\dot{r}\\
&p_{\beta}=\PDy{\dot{\beta}}{L}=\mu r^2\dot{\beta}\\
&p_{\lambda}=\PDy{\dot{\lambda}}{L}=\mu r^2\cos{\beta}^2\dot{\lambda}\ \dot{p}_{\lambda}=0
\end{align*}

\end{block}

\begin{block}{Hamiltoniana}

\begin{align*}
&H(p,q)=\frac{1}{2\mu}(p_r^2+\frac{1}{r^2}p_{\beta}^2+\frac{1}{r^2\cos{\beta}^2}p_{\lambda}^2)-\frac{k^2\mu}{r}
\end{align*}


\end{block}

\end{frame}

\begin{wordonframe}{Meccanica Hamiltoniana}

\begin{block}{Equazioni di Hamilton}
\begin{align*}
&\dot{q}=\PDy{p}{H}\\
&\dot{p}=-\PDy{q}{H}
\end{align*}
\end{block}

$\TDy{t}{H}=\PDy{t}{H}$

\end{wordonframe}


\section{Equazione di Hamilton-Jacobi}

\begin{block}{Trasformazioni canoniche}

\begin{align*}
(p,q)\to(\xi(p,q),\eta(p,q))
\end{align*}

che lasciano invariate le equazioni di Hamilton.

\end{block}

\begin{block}{Problema di H-J}

Trovare trasformazione canonica tale che $K=0$: $(\xi,\eta)$ sono costanti del moto.

\end{block}

\begin{block}{Funzione generatrice $S(q,\eta,t)$}

\begin{align*}
&\xi=\PDof{\eta}S(q,\eta,t)\ (2: \xi(p,q))\\
&p=\PDof{q}S(q,\eta,t)\ (1:\eta(q,p))\\
&K=H+\PDy{t}{S}\ K=0\Leftrightarrow H(q,\PDy{q}{S},t)=-\PDy{t}{S}
\end{align*}


\end{block}


\section{Sistemi integrabili e degenerazione}


